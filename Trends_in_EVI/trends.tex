%!TEX program =pdflatex
\documentclass{beamer}
\usetheme{CambridgeUS}
%%define new comand
\def\argmin{\mathop{\rm arg~min}\limits}
\def\argmin{\mathop{\rm arg~min}\limits}
\newcommand{\bdelta}{\boldsymbol{\delta}}
\newcommand{\bbeta}{\boldsymbol{\beta}}
\newcommand{\bSigma}{\boldsymbol{\Sigma}}
\newcommand{\brho}{\displaystyle{\large{\boldsymbol{\rho}}}}
\newcommand{\bgamma}{\boldsymbol{\gamma}}
\newcommand{\bfeta}{\boldsymbol{\eta}}
\newcommand{\bPsi}{\boldsymbol{\Psi}}
\newcommand{\bmu}{\boldsymbol{\mu}}
\newcommand{\bvartheta}{\boldsymbol{\vartheta}}
\newcommand{\bzero}{\mathbf{0}}
\newcommand{\bone}{\mathbf{1}}
\newcommand{\bA}{\mathbf{A}}
\newcommand{\ba}{\mathbf{a}}
\newcommand{\bB}{\mathbf{B}}
\newcommand{\bb}{\mathbf{b}}
\newcommand{\bD}{\mathbf{D}}
\newcommand{\bU}{\mathbf{U}}
\newcommand{\bu}{\mathbf{u}}
\newcommand{\bV}{\mathbf{V}}
\newcommand{\bW}{\mathbf{W}}
\newcommand{\bw}{\mathbf{w}}
\newcommand{\bX}{\mathbf{X}}
\newcommand{\bx}{\mathbf{x}}
\newcommand{\bY}{\mathbf{Y}}
\newcommand{\by}{\mathbf{y}}
\newcommand{\bZ}{\mathbf{Z}}
\newcommand{\bz}{\mathbf{z}}
\newcommand{\suit}[1]{\left(#1\right)}
\newcommand{\abs}[1]{\left\vert#1\right\vert}
\newcommand{\set}[1]{\left\{#1\right\}}
\newcommand{\msuit}[1]{\left[ #1 \right]}
\author{Laurens de Haan and Chen Zhou }
\date{Dec, 07, 2020}
\title{Trends in Extreme Value Index}
\begin{document}
\begin{frame}
\titlepage
\begin{center}
    Published on Journal of the American Statistical Association(2020).

    \bigskip
    Presented by Liujun Chen

\end{center}
\end{frame}


\AtBeginSection[]
{
    \begin{frame}
        \frametitle{Table of Contents}
        \tableofcontents[currentsection]
    \end{frame}
}


\section{Introduction}

\begin{frame}
    \frametitle{Background}
\begin{itemize}
    \item Classic extreme value analysis assumes that the observations are i.i.d.
    \bigskip
    \item Recent studies aim at dealing with the case when observations are drawn from different distributions.
    \bigskip
    \item This paper considers a continuously changing extreme value index and try to estimate the functional extreme value index accurately.
\end{itemize}
\end{frame}

\begin{frame}
    \frametitle{Model Setting}
\begin{itemize}
    \item Consider a set of distributions $F_s(x)$ for $s\in [0,1]$ and independent random variables $X_i\sim F_{\frac{i}{n}}(x)$ for $i=1,\dots,n$.
    \medskip
    \item Here $F_s(x)$ is assumed to be continuous with respect to $s$ and $x$. And assume that $F_s \in D_{\gamma(s)}$.
    \medskip
    \item This article considers the case that the function $\gamma$ is positive and continuous on $[0,1]$.
    \medskip
    \item The goal is to estimate the function $\gamma$ and test the hypothesis that $\gamma=\gamma_0$ for some given function $\gamma_0$.
\end{itemize}

    

\end{frame}



\section{Methodology}


\section{Testing Trends}

\section{Application}


\end{document}