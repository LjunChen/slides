%!TEX program =pdflatex
\documentclass{beamer}
\usetheme{CambridgeUS}
%%define new comand
\def\argmin{\mathop{\rm arg~min}\limits}
\def\argmin{\mathop{\rm arg~min}\limits}
\newcommand{\bdelta}{\boldsymbol{\delta}}
\newcommand{\bbeta}{\boldsymbol{\beta}}
\newcommand{\bSigma}{\boldsymbol{\Sigma}}
\newcommand{\brho}{\displaystyle{\large{\boldsymbol{\rho}}}}
\newcommand{\bgamma}{\boldsymbol{\gamma}}
\newcommand{\bfeta}{\boldsymbol{\eta}}
\newcommand{\bPsi}{\boldsymbol{\Psi}}
\newcommand{\bmu}{\boldsymbol{\mu}}
\newcommand{\bvartheta}{\boldsymbol{\vartheta}}
\newcommand{\bzero}{\mathbf{0}}
\newcommand{\bone}{\mathbf{1}}
\newcommand{\bA}{\mathbf{A}}
\newcommand{\ba}{\mathbf{a}}
\newcommand{\bB}{\mathbf{B}}
\newcommand{\bb}{\mathbf{b}}
\newcommand{\bD}{\mathbf{D}}
\newcommand{\bU}{\mathbf{U}}
\newcommand{\bu}{\mathbf{u}}
\newcommand{\bV}{\mathbf{V}}
\newcommand{\bW}{\mathbf{W}}
\newcommand{\bw}{\mathbf{w}}
\newcommand{\bX}{\mathbf{X}}
\newcommand{\bx}{\mathbf{x}}
\newcommand{\bY}{\mathbf{Y}}
\newcommand{\by}{\mathbf{y}}
\newcommand{\bZ}{\mathbf{Z}}
\newcommand{\bz}{\mathbf{z}}
\newcommand{\suit}[1]{\left(#1\right)}
\newcommand{\abs}[1]{\left\vert#1\right\vert}
\newcommand{\set}[1]{\left\{#1\right\}}
\newcommand{\msuit}[1]{\left[ #1 \right]}
\author{Laurens de Haan and Chen Zhou }
\date{Dec, 07, 2020}
\title{Trends in Extreme Value Index}
\begin{document}
\begin{frame}
\titlepage
\begin{center}
    Published on Journal of the American Statistical Association(2020).

    \bigskip
    Presented by Liujun Chen

\end{center}
\end{frame}


\AtBeginSection[]
{
    \begin{frame}
        \frametitle{Table of Contents}
        \tableofcontents[currentsection]
    \end{frame}
}


\section{Introduction}

\begin{frame}
    \frametitle{Background}
\begin{itemize}
    \item Classic extreme value analysis assumes that the observations are i.i.d.
    \bigskip
    \item Recent studies aim at dealing with the case when observations are drawn from different distributions.
    \bigskip
    \item This paper considers a continuously changing extreme value index and try to estimate the functional extreme value index accurately.
\end{itemize}
\end{frame}

\begin{frame}
    \frametitle{Model Setting}
\begin{itemize}
    \item Consider a set of distributions $F_s(x)$ for $s\in [0,1]$ and independent random variables $X_i\sim F_{\frac{i}{n}}(x)$ for $i=1,\dots,n$.
    \medskip
    \item Here $F_s(x)$ is assumed to be continuous with respect to $s$ and $x$. And assume that $F_s \in D_{\gamma(s)}$.
    \medskip
    \item This article considers the case that the function $\gamma$ is positive and continuous on $[0,1]$.
    \medskip
    \item The goal is to estimate the function $\gamma$ and test the hypothesis that $\gamma=\gamma_0$ for some given function $\gamma_0$.
\end{itemize}

\end{frame}



\section{Methodology and Main Results}

\begin{frame}
    \frametitle{Methodology}
\begin{itemize}
    \item The idea for estimating $\gamma(s)$ locally is similar to the kernel density estimation.
    \medskip
    \item More specifically, use only observations $X_i$ in the $h$-neighborhood of $s$, that is 
    $$
i\in I_n(s)=\set{\abs{\frac{i}{n}-s}\le h},
    $$
    where $h:=h(n)$ is the bandwidth such that as $n\to \infty$, $h \to \infty$ and $nh\to \infty$.
    \medskip
\item Correspondingly, there will be $[2nh]$ observations for $s\in [h,1-h]$.
\end{itemize}
\end{frame}

\begin{frame}
    \frametitle{Methodology}
\begin{itemize}
    \item To apply any known estimators for extreme value index, such as Hill estimator, choose am intermediate sequence $k=k(n)$ such that $k\to \infty, k/n \to 0$ as $n \to \infty$.
    \item Then one can use the top $[2kh]$ order statistics among the $[2nh]$ local observations in the $h$-neighborhood of $s$ to estimate $\gamma(s)$.
    \item The local Hill estimator for $\gamma(s)$ is defined as follows. Rank the $[2nh]$ observations $X_i$ with $i \in I_n(s)$ as $X_{1,[2nh]}^{(s)}\le \cdots \le X_{[2nh],[2nh]}^{(s)}$. Then
    $$
    \hat{\gamma}_{H}(s):=\frac{1}{[2 k h]} \sum_{i \in I_{n}(s)}\left(\log X_{i}-\log X_{[2 n h]-[2 k h],[2 n h]}\right)^{+}.
    $$
\end{itemize}
\end{frame}






\begin{frame}
    \frametitle{Second order condition}
To obtain the asymptotic theory, the following conditions are required.
    

\end{frame}

\section{Testing Trends}

\section{Application}


\end{document}