%!TEX program =pdflatex
\documentclass{beamer}
\usetheme{CambridgeUS}
%%define new comand
\def\argmin{\mathop{\rm arg~min}\limits}
\def\argmin{\mathop{\rm arg~min}\limits}
\newcommand{\bdelta}{\boldsymbol{\delta}}
\newcommand{\bbeta}{\boldsymbol{\beta}}
\newcommand{\bSigma}{\boldsymbol{\Sigma}}
\newcommand{\brho}{\displaystyle{\large{\boldsymbol{\rho}}}}
\newcommand{\bgamma}{\boldsymbol{\gamma}}
\newcommand{\bfeta}{\boldsymbol{\eta}}
\newcommand{\bPsi}{\boldsymbol{\Psi}}
\newcommand{\bmu}{\boldsymbol{\mu}}
\newcommand{\bvartheta}{\boldsymbol{\vartheta}}
\newcommand{\bzero}{\mathbf{0}}
\newcommand{\bone}{\mathbf{1}}
\newcommand{\bA}{\mathbf{A}}
\newcommand{\ba}{\mathbf{a}}
\newcommand{\bB}{\mathbf{B}}
\newcommand{\bb}{\mathbf{b}}
\newcommand{\bD}{\mathbf{D}}
\newcommand{\bU}{\mathbf{U}}
\newcommand{\bu}{\mathbf{u}}
\newcommand{\bV}{\mathbf{V}}
\newcommand{\bW}{\mathbf{W}}
\newcommand{\bw}{\mathbf{w}}
\newcommand{\bX}{\mathbf{X}}
\newcommand{\bx}{\mathbf{x}}
\newcommand{\bY}{\mathbf{Y}}
\newcommand{\by}{\mathbf{y}}
\newcommand{\bZ}{\mathbf{Z}}
\newcommand{\bz}{\mathbf{z}}
\newcommand{\suit}[1]{\left(#1\right)}
\newcommand{\abs}[1]{\left\vert#1\right\vert}
\newcommand{\set}[1]{\left\{#1\right\}}
\newcommand{\msuit}[1]{\left[ #1 \right]}
\author{Simpson, Wadsworth and Tawn (2020)}
\title{Determining the dependence structure of multivariate extremes}
\begin{document}
\begin{frame}
\titlepage
\begin{center}
    Published on Biometrika(2020). 
    \bigskip

    Presented by Liujun Chen.
\end{center}
\end{frame}



\begin{frame}
    \frametitle{Introduction}
\begin{itemize}
    \item When modelling in multivariate extreme value analysis, one often needs to exploit extremal dependence features.
    \item Consider the random vector $X=(X_1,\dots,X_d)$, with $X_i\sim F_i$.
    \item Consider a subset of these variables $X_C=\set{X_i:i\in C}$, $C$ lies in the power set of $D=\set{1,\dots,d}$.
    \item For any $C$ with $\abs{C}\ge 2$, extremal dependence with $X_C$ can be summarized by 
    $$
\chi_C=\lim_{u\to 1} pr \set{F_i(X_i)>u : i\in C}/(1-u)
    $$
    if the limit exists.
\end{itemize}
    

\end{frame}


\begin{frame}
    \frametitle{Introduction}
    \begin{itemize}
        \item If $\chi_C>0$, the variables in $X_C$ are asymptotically dependent (they can take their largest values simultaneously).
        \bigskip
        \item  If $\chi_C=0$, the variables in $X_C$ cannot all take their largest values simultaneously.
        \bigskip
        \item However, it is possible that for some $\underline{C} \subset C, \chi_{\underline{C}}>0$. 
    \end{itemize}
\end{frame}


\begin{frame}
    \frametitle{Introduction}
    Many models for multivariate extremes are applicable only when
    \begin{itemize}
        \item  data exhibit either full asymptotical dependence, entailing $\chi_C>0$ for all $C\in z^D \backslash \emptyset$ with $\abs{C}\ge 2$.
        \item or full asymptotical independence $\chi_{i,j}=0$ for all $i<j$.
    \end{itemize}
\bigskip

However, it is often the case that some $\chi_C$ are positive while others are zero. 

\bigskip

The extremal dependence between variables can thus have a complicated structure.

    

\end{frame}
\end{document}